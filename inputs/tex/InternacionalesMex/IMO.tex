En 1959 Rumania organizó la Primera Olimpiada Internacional de Matemáticas con la participación de sólo 7 países: Hungría, la URSS, Bulgaria, Polonia, Checoslovaquia, la República Democrática Alemana y Rumania. A partir de entonces la Olimpiada Internacional se celebra año con año (casi siempre en julio) con la participación de países de los cinco continentes.

Los ganadores del primer Concurso Nacional asistieron a la 29\textsuperscript{a} Olimpiada Internacional de Matemáticas, celebrada en Canberra, Australia, en julio de 1988; a partir de ese año México ha asistido a la emisión anual de la Olimpiada Internacional de Matemáticas. México organizó la 46\textsuperscript{a} Olimpiada Internacional en Mérida, Yucatán en julio de 2005.

Los resultados de las delegaciones mexicanas en la Olimpiada Internacional han sido: