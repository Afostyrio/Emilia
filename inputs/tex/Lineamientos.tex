\section{Estructura y lineamientos}
\begin{lineamientos}
    \item Aspectos generales y objetivos
    \begin{lineamientos}
        \item La Olimpiada Mexicana de Matemáticas (OMM) es un programa de la Sociedad Matemática Mexicana. Su objetivo principal es el de fomentar y estimular el estudio de las matemáticas como una disciplina del pensamiento que desarrolla la inteligencia del estudiante mediante métodos de razonamiento estructurado, deductivo y creativo.
        \item El programa básico de la OMM se desarrolla anualmente en cuatro
        etapas:
        \begin{itemize}[label=$-$]
            \item los Concursos Estatales,
            \item el Concurso Nacional,
            \item el entrenamiento y la selección de las delegaciones que representarán a México en olimpiadas internacionales, y
            \item la participación en olimpiadas internacionales.            
        \end{itemize}
        \item La organización general de la OMM está a cargo de un Comité Organizador.
    \end{lineamientos}
    \item Estructura en la organización
    \begin{lineamientos}
        \item El Comité Organizador de la Olimpiada Mexicana de Matemáticas está formado por:
        \begin{itemize}[label=$-$]
            \item el presidente de la OMM, y
            \item los miembros.
        \end{itemize}
        \item En cada uno de los estados de la República y en el Distrito Federal (en lo sucesivo, denominado también estado) se nombra un delegado.
        \item Tanto los miembros del Comité como los delegados son miembros de la Sociedad Matemática Mexicana durante el periodo de su cargo.
        \item El Comité trabaja por medio de comisiones que se encargan de alguna tarea específica. Cada comisión está integrada por un coordinador y los miembros que sean necesarios para el buen funcionamiento de ella.
        \item Los antiguos presidentes de la OMM, junto con un miembro del Consejo Consultivo de la Sociedad Matemática Mexicana (SMM) y el presidente de la SMM forman el Consejo Consultivo de la OMM.
    \end{lineamientos}
    \item Designación de los miembros del Comité Organizador y de los delegados de la OMM
    \begin{lineamientos}
        \item El presidente de la OMM se elige por votación escrita entre los socios de la SMM que estén al corriente en su pago de cuotas. Los candidatos deben presentar un resumen curricular y un programa de trabajo por escrito en las fechas que indique la convocatoria de la Sociedad Matemática Mexicana. Ésta debe publicar estos programas durante el periodo de votaciones (antes del Congreso Nacional de la SMM). Los candidatos deben ser miembros de la comunidad matemática del país y deben haber demostrado ampliamente su interés por la OMM en el pasado. Los resultados de la votación se publican durante el Congreso Nacional de la SMM. Un candidato se considera electo si al menos la décima parte de la población mencionada vota, y la mayoría de los votos emitidos son votos a favor de dicho candidato. En caso de no reunirse este mínimo de votos, se elige al presidente por mayoría de votos durante la Asamblea General del Congreso Nacional de la SMM. El presidente entra en funciones el 1\({}^\circ\) de febrero del año siguiente al de su elección. Dura en su cargo 4 años. El presidente puede ser reelegido. Si por algún motivo el presidente no puede ejercer su cargo a término, se designa otro mediante elección por la SMM por el periodo que le falte al presidente en turno.
        \item El presidente propone a la Junta Directiva de la SMM los miembros del Comité Organizador. Si la Junta Directiva los ratifica, inician sus funciones el mismo día que el presidente o a partir de que éste haga la proposición correspondiente.
        \item Cada miembro del Comité puede tener a su cargo una comisión específica y designar, junto con el presidente del Comité Organizador de la OMM, a los integrantes de su comisión.
        \item El presidente de la OMM designa anualmente a los delegados estatales
    \end{lineamientos}
    \item Funciones del Comité Organizador, de los delegados y del Consejo Consultivo de la OMM
    \begin{lineamientos}
        \item El Comité Organizador de la OMM está encargado de vigilar el cumplimiento de estos lineamientos y de llevar a cabo el programa general de la Olimpiada de Matemáticas en México, que incluye los siguientes puntos:
        \begin{itemize}[label=$-$]
            \item difusión,
            \item organización del Concurso Nacional,
            \item entrenamientos de los seleccionados nacionales, y
            \item representación de nuestro país en olimpiadas internacionales.
        \end{itemize}
        \item Cada delegado estatal es responsable de la organización de la Olimpiada de Matemáticas en su estado, ajustándose a la filosofía y lineamientos de la OMM, con el objeto de enviar un equipo que represente a su estado en el Concurso Nacional.
        \item El Consejo Consultivo de la OMM debe:
        \begin{itemize}[label=$-$]
            \item vigilar el buen funcionamiento del programa de la OMM,
            \item revisar que los candidatos a presidente de la OMM cumplan con los requisitos del puesto,
            \item apoyar al presidente de la OMM en la búsqueda de los recursos económicos necesarios para el buen funcionamiento del programa de la Olimpiada.
        \end{itemize}
    \end{lineamientos}
    \item Concursos Estatales

    La organización y mecanismos de participación en cada Concurso Estatal son responsabilidad del Comité Estatal correspondiente, apegándose al espíritu general del Concurso Nacional
    \item Concurso Nacional

    El Concurso Nacional se lleva a cabo durante una semana (usualmente en el mes de noviembre) en algún estado de la República elegido por el Comité Organizador de la OMM.
    
    Durante la semana de celebración del Concurso Nacional se lleva a cabo el examen, las sesiones de coordinación, las reuniones del jurado y la ceremonia de premiación, además de diversas actividades sociales y culturales para los participantes.

    \item Forma de participación en el Concurso Nacional
    \begin{lineamientos}
        \item Al Concurso Nacional de la OMM están invitados todos los estados de la República Mexicana.
        \item Cada estado tiene derecho de participar en el Concurso Nacional de la OMM hasta con seis estudiantes, acompañados por un profesor (o delegado). El Distrito Federal puede participar hasta con diez alumnos (y dos profesores). Al Concurso Nacional se invita también a un observador del estado sede del Concurso Nacional del año siguiente.
        \item La participación es individual y gratuita.
        \item Cada estudiante concursante debe satisfacer lo siguiente:
        \begin{lineamientos}
            \item No cumplir 20 años antes del concurso de la Olimpiada Internacional en la cual participaría si resultara ganador. (Dicho concurso se lleva a cabo usualmente en julio del año siguiente a la celebración del Concurso Nacional; la fecha exacta se da a conocer con anticipación en la propaganda respectiva.)
            \item Estar inscrito en el bachillerato (o equivalente) o en algún grado inferior durante la celebración del Concurso Nacional.
            \item Garantizar que no estará inscrito en ninguna universidad o equivalente durante la Olimpiada Internacional que sucede al Concurso Nacional.    
        \end{lineamientos}
    \end{lineamientos}
    \item Examen del Concurso Nacional de la OMM
    \begin{lineamientos}
        \item El examen que se aplica a los alumnos participantes en el Concurso Nacional de la OMM consta de dos pruebas escritas, cada una con una duración de cuatro horas y media, realizadas en dos días distintos al iniciar la semana del Concurso Nacional.
        \item Cada prueba consta de tres problemas de matemáticas. Cada concursante presenta por escrito su solución a dichos problemas.
        \item Los concursantes no deben usar libros, libretas de apuntes, calculadoras, ni tablas de ningún tipo durante el examen. Deben además sujetarse a las instrucciones específicas del examen, según se les haya indicado previamente.
    \end{lineamientos}
    \item Tipo de problemas en el examen del Concurso Nacional
    \begin{lineamientos}
        \item Los problemas del examen del Concurso Nacional versan sobre distintos temas de matemáticas básicas (previos a Geometría Analítica, sin incluir ésta). La resolución correcta de los problemas del examen requiere, en general, de mucho ingenio y de gran habilidad en el manejo de esos conocimentos básicos de matemáticas.
        \item El Comité Organizador de la OMM elabora el examen con base en los problemas que le envían las delegaciones estatales, así como miembros de la comunidad matemática del país.
    \end{lineamientos}
    \item Jurado del Concurso Nacional de la OMM
    \begin{lineamientos}
        \item El Jurado del Concurso Nacional está integrado por los delegados de los estados (o los profesores que los representan durante el Concurso Nacional) y por tres miembros designados por el Comité Organizador de la OMM, uno de los cuales preside el Jurado.
        \item Son funciones del Jurado:
        \begin{lineamientos}
            \item Decidir sobre posibles respuestas a las preguntas que, sobre los enunciados de los problemas, formulen los concursantes durante la primera hora de la prueba.
            \item Establecer, junto con el Tribunal de Coordinación, las pautas para la calificación de soluciones parciales en los problemas del examen.
            \item Tomar decisiones en caso de que se presente diferencia de opinión entre el Tribunal de Coordinación y el delegado de algún estado sobre la calificación de su alumno.
            \item Decidir sobre el otorgamiento de premios especiales y ratificar la distribución de premios según los lineamientos correspondientes.
        \end{lineamientos}
        \item En las reuniones del Jurado, cada miembro, con excepción del presidente, tiene derecho a un voto. En caso de empate, el presidente del Jurado tiene voto dirimente.
        \item A las reuniones del Jurado pueden asistir como observadores los miembros del Comité Organizador de la OMM y un profesor más por cada estado, si el delegado así lo decide. Con autorización del mismo Jurado, pueden asistir otras personas, pero sólo el Jurado y los observadores pueden participar en las discusiones del Jurado.
    \end{lineamientos}
    \item Calificación del examen del Concurso Nacional de la OMM
    \begin{lineamientos}
        \item El Comité Organizador de la OMM designa un Tribunal de Coordinación que se divide en seis equipos (uno para cada problema). El Tribunal de Coordinación tiene un Jefe nombrado por el Comité Organizador de la OMM.
        \item Cada equipo del Tribunal de Coordinación presenta al Jurado una propuesta de puntaje para la calificación del problema que va a coordinar. Con base en los comentarios del Jurado y a su propia evaluación de las posibles soluciones de los concursantes, determina las pautas de calificación. Las calificaciones son enteros del 0 al 7.
        \item Los exámenes resueltos por los estudiantes se hacen llegar al Tribunal de Coordinación. Los delegados reciben también una copia de los exámenes resueltos por sus respectivos alumnos.
        \item Cada delegado califica los exámenes de sus alumnos siguiendo los criterios acordados en la reunión correspondiente. A su vez, los equipos del Tribunal de Coordinación revisan los exámenes y deciden sobre posibles agregados a los criterios de puntuación, según las soluciones que hubieran presentado algunos alumnos y que no hubieran sido contempladas antes de ver los exámenes. Estos agregados se informan claramente a todos los delegados.
        \item En la calificación de las pruebas, el texto presentado por los estudiantes debe ser preservado de cualquier alteración.
        \item Durante la semana en que se celebra el Concurso Nacional, el Comité Organizador de la OMM establece un calendario de coordinaciones en el cual cada delegado presenta, ante el equipo de coordinación correspondiente, una evaluación fundamentada de la solución de cada uno de sus estudiantes. El equipo de coordinación del problema en cuestión determina la calificación respectiva. Si el delegado no está de acuerdo sobre alguna de sus calificaciones, se pide la intervención del Jefe del Tribunal. En caso de mantenerse el desacuerdo, éste se lleva ante el Jurado del Concurso Nacional, el cual da su veredicto final.
    \end{lineamientos}
    \item Premiación en el Concurso Nacional de la OMM
    \begin{lineamientos}
        \item Se otorgan primeros, segundos y terceros lugares. Éstos se asientan en un diploma.
        \item En conjunto, el número de primeros, segundos y terceros lugares es aproximadamente igual al cincuenta por ciento del total de los participantes, y la razón entre primeros, segundos y terceros lugares es aproximadamente igual a 1:2:3. Para determinar exactamente el número de alumnos premiados en cada lugar se hace lo siguiente:

        Se ponen en una lista en orden decreciente todas las calificaciones de los alumnos, incluyendo repeticiones. Se otorga un primer lugar a todos los alumnos que tengan una calificación igual o superior a la calificación del alumno que aparece en posición 16. Los segundos lugares se determinan buscando la última calificación que aparece en el cuarto superior de la lista, y se le otorga segundo lugar a todos los alumnos que, no habiendo obtenido primer lugar, tienen una calificación igual o superior a esa puntuación. Para determinar los alumnos con tercer lugar se hace lo análogo que con los segundos lugares, pero buscando la última puntuación en la mitad superior de la tabla de calificaciones. (Nota: En caso de que el número de alumnos no sea divisible por 2 o por 4, se toma la parte entera de la división; por ejemplo, si hay en total 191 participantes, entonces la mitad superior comprende 95 
        alumnos y el cuarto superior comprende 47).
        \item Se otorgan menciones honoríficas a los alumnos que no obtengan un primer, segundo o tercer lugar, pero que obtengan el máximo puntaje (7 puntos) en al menos un problema del examen.
        \item Se pueden otorgar premios especiales a aquellas soluciones presentadas por los alumnos en algún problema del examen si, a juicio del Tribunal de Coordinación, éstas son muy sobresalientes. Se entrega también un diploma especial a los alumnos que obtengan la mejor puntuación en el examen.
        \item Dentro del Concurso Nacional se selecciona también un grupo de alumnos, candidatos a participar en la Olimpiada Centroamericana y del Caribe del año siguiente. Estos alumnos se seleccionan de entre los alumnos con mejores puntuaciones en el Concurso Nacional que cumplan 16 años en una fecha posterior al 31 de diciembre del año de celebración del Concurso, y que todavía puedan participar en el Concurso Nacional del año siguiente. El número de alumnos seleccionados se determina como el menor número que satisfaga las dos condiciones siguientes simultáneamente: debe haber 3 alumnos seleccionados fuera del grupo de los ocupantes de los primeros lugares y debe haber al menos 5 alumnos en el grupo seleccionado.
        \item El estado sede del Concurso Nacional entrega el Premio Superación del año a la delegación que muestre progreso relativo mayor, según los lineamientos indicados en el Anexo. También se da un diploma al segundo y tercer lugares en esta competencia.
        \item Cada concursante recibe un diploma que acredita su participación en el Concurso Nacional de la OMM.
        \item Los premios y diplomas se entregan en el acto de clausura del Concurso Nacional de la OMM.
    \end{lineamientos}
    \item Selección y entrenamientos de las delegaciones mexicanas
    \begin{lineamientos}
        \item Dentro del grupo de primeros lugares se selecciona un equipo de máximo 6 alumnos el cual representa a México en la Olimpiada Internacional que sucede al Concurso Nacional (que se lleva a cabo generalmente en julio del año siguiente a la celebración del Concurso Nacional). La selección se realiza  exámenes eliminatorios sucesivos elaborados por el Comité Organizador de la OMM (que se aplican durante los entrenamientos) y un examen definitivo (que se aplica a más tardar en mayo).
        \item Para conformar la delegación que representa a México en la Olimpiada Iberoamericana del año siguiente al Concurso Nacional (generalmente celebrada en septiembre) se hace lo siguiente. En el examen de selección de mayo que se aplica a los ganadores del primer lugar del Concurso Nacional, se escoge a los 6 alumnos con mejor puntaje de entre los que satisfacen los requisitos de participación en la Olimpiada Iberoamericana (cumplir 19 años en una fecha posterior al 31 de diciembre del año de su celebración, y no haber participado antes en dos Olimpiadas Iberoamericanas); también tienen derecho de participar en ese examen los alumnos ganadores de primer lugar en el Concurso Nacional del año anterior que satisfacen los requisitos de participación de la Olimpiada Iberoamericana, pero que no hubieran participado en el Concurso Nacional del año. A lo más dos alumnos pueden integrarse a partir de ese momento a la preselección, agregándose a los 6 alumnos ya seleccionados, siempre y cuando obtengan una calificación igual o superior al sexto alumno del grupo de ganadores del año. En agosto se hace la selección definitiva de a lo más 4 alumnos.
        \item Los alumnos seleccionados en el Concurso Nacional como candidatos a participar en la Olimpiada Centroamericana y del Caribe del año siguiente (que se celebra generalmente en julio) presentan en mayo un examen. A lo más 3 alumnos con mayor puntaje en ese examen representan a México en el concurso.
        \item Todos los alumnos preseleccionados y seleccionados reciben entrenamientos especiales (aproximadamente una semana al mes) dirigidos por el Comité Organizador de la OMM. Estos entrenamientos tienen el propósito de prepararlos para representar a nuestro país en las olimpiadas internacionales de matemáticas correspondientes.
        \item En todo momento de su participación, los alumnos preseleccionados deben observar una conducta aceptable de respeto y compañerismo. El Comité Organizador de la OMM podrá suspender a cualquier alumno que no cumpla con esto.
    \end{lineamientos}
    \item Otras actividades de la OMM
    \begin{lineamientos}
        \item A lo largo del año el Comité Organizador de la OMM promueve la visita de profesores que imparten cursos de matemáticas de tipo olímpico a diferentes estados del país, y colabora en la elaboración de exámenes estatales en sus distintas fases con los estados que así lo solicitan.
        \item El Comité Organizador de la OMM organiza también un curso anual para entrenadores de las Olimpiadas de Matemáticas; dicho curso se lleva a cabo durante un fin de semana cerca de la semana santa.
    \end{lineamientos}
    \item Otras consideraciones
    \begin{lineamientos}
        \item Cualquier duda de interpretación, situación no recogida en estos lineamientos de la OMM, o asunto especial, debe ser decidido por:
        \begin{lineamientos}
            \item el Jurado del Concurso Nacional, si se trata de una situación particular que se presente durante el concurso,
            \item el Comité Organizador de la OMM, si se trata de una situación general de organización o de procedimiento.
        \end{lineamientos}
        \item La modificación de cualquier práctica según la descripción aquí presentada, así como la inclusión de nuevas prácticas debe ser sugerida al Comité Organizador de la OMM. Si éste lo considera pertinente, pasará la propuesta correspondiente a todos los delegados por escrito. La decisión de cambio se hará si la mayoría de los delegados lo aceptan.
    \end{lineamientos}
\end{lineamientos}

\section{Anexo}
\subsection{Reglamento del concurso de la Copa Superación en el Concurso Nacional de la Olimpiada Mexicana de Matemáticas}
Pueden competir por la Copa Superación todos los estados que hayan participado en por lo menos dos de los últimos tres Concursos Nacionales anteriores (con cualquier número de alumnos) y que participen con equipo completo ese año. Se premia a los primeros tres lugares de acuerdo con el mayor puntaje de progreso relativo, el cual se calcula de la manera siguiente:
\begin{enumerate}[label=\arabic*.]
    \item Anualmente se calcula el promedio general de calificaciones de todos los alumnos participantes en el año.
    \item Se obtiene el promedio anual de cada equipo (suma de las calificaciones de los alumnos que integren la delegación, dividida entre el número de integrantes del equipo), y se divide entre el promedio general anual correspondiente. Este promedio se multiplica por 100. Al número obtenido se le llama promedio normalizado del equipo en el año.
    \item Se calcula el promedio de los dos últimos años de participación de cada equipo (suma de los dos promedios normalizados obtenidos durante los dos últimos años de participación, dividida entre 2).
    \item El progreso relativo de cada equipo es la diferencia del promedio normalizado del año menos 1.1 veces el promedio en los dos últimos años de participación.
\end{enumerate}
La fórmula ha sido obtenida considerando lo siguiente:
\begin{lineamientos}
    \item El propósito de la Copa Superación es impulsar el progreso de los equipos, tomando en cuenta que las altas calificaciones absolutas son premiadas de manera regular durante el concurso. Así, el factor 1.1 del inciso (4) tiene el efecto de dar mayor valor a una diferencia de crecimiento a los equipos con puntuaciones más bajas (por ejemplo, la fórmula considera que un equipo que aumenta su promedio de 150 a 200 tiene menor progreso relativo que un equipo que aumenta su promedio de 100 a 150.)
    \item Los problemas propuestos en una Olimpiada determinada pueden ser más difíciles que en las anteriores. Al normalizar (dividir entre los promedios generales de calificaciones en los años correspondientes) se elimina la posibilidad de que estados cuyas bajas calificaciones no dependan de la prueba (por ejemplo, que mantengan una constante de 0) estén por encima de otros estados que hayan trabajado relativamente mejor que en los años anteriores (por ejemplo, estados que obtengan puntuaciones negativas a causa de la mayor dificultad del examen).
    \item El factor 100 del inciso (2) tiene el propósito de no trabajar con demasiadas cifras decimales.
\end{lineamientos}
En caso de empates se toman en cuenta, en orden sucesivo, los siguientes puntos:
\begin{lineamientos}
    \item En caso de que alguno de los equipos empatados no hubiera tenido participación con equipo completo en los años anteriores (los que entraron en juego al aplicar la fórmula), se le da ventaja al equipo con mayor participación (este número se obtiene como el cociente del número de alumnos que hubieran participado, entre el número de alumnos que deberían haberlo hecho).
    \item En caso que persista el empate, tiene ventaja el equipo con progreso absoluto mayor (es decir, sin considerar el factor 1.1 del inciso (4)).
    \item En caso que persista el empate, gana el equipo con mayor promedio en el año de competencia por la Copa.
\end{lineamientos}