\section{Etapas}
La Olimpiada Mexicana de Matemáticas consta de tres etapas:
\begin{enumerate}
    \item los Concursos Estatales,
    \item el Concurso Nacional, y
    \item el entrenamiento y selección de las delegaciones que representarán a México en los concursos internacionales.
\end{enumerate}

\subsection{Concursos Estatales}
La primera etapa de la Olimpiada Mexicana de Matemáticas está formada por los Concursos Estatales. La organización y mecanismos de participación en cada uno de ellos es responsabilidad del Comité Estatal correspondiente, apegándose al espíritu general del Concurso Nacional. De estos concursos saldrán los integrantes de la delegación que representará a su estado en el Concurso Nacional de la Olimpiada Mexicana de Matemáticas

\subsection{Concurso Nacional}
El Concurso Nacional se lleva a cabo durante una semana, usualmente en el mes de noviembre, en algún estado de la República elegido por el Comité Organizador de la OMM. Durante esa semana se realizan el examen, las sesiones de coordinación, las reuniones del Jurado y la ceremonia de premiación, además de diversas actividades sociales y culturales para los participantes.

Al Concurso Nacional de la OMM están invitados todos los estados de la República Mexicana. Cada uno tiene derecho de participar hasta con seis estudiantes, por hasta cuatro profesores. Los alumnos participantes son los ganadores de los Concursos Estatales correspondientes.

El sistema de competencia y evaluación de la Olimpiada Mexicana de Matemáticas sigue en gran medida el modelo de la Olimpiada Internacional.

El Comité Organizador de la OMM elabora el examen a partir de los problemas que le envían las delegaciones estatales, así como miembros de la comunidad  del país. Los problemas elegidos versan sobre distintos temas de matemáticas básicas y deben ser inéditos. El examen consta de dos pruebas escritas que se aplican en dos días consecutivos, cada una de las cuales consta de tres problemas y se otorgan 4 horas y media para su resolución. Cada concursante presenta por escrito su solución para cada uno de los seis problemas. La resolución correcta de los problemas del examen requiere, en general, de mucho ingenio y de gran habilidad en el manejo de conocimentos básicos de matemáticas.

Los exámenes resueltos por los estudiantes se hacen llegar a un Tribunal de Coordinación dividido en seis equipos (uno para cada problema). Los delegados reciben también una copia de los exámenes resueltos por sus respectivos alumnos. Cada delegado califica los exámenes siguiendo los criterios acordados en la reunión, mediante los cuales se siguen pautas para otorgar puntajes (enteros del 0 al 7) a las posibles soluciones de los problemas. A su vez, los equipos del Tribunal de Coordinación revisan los exámenes. Durante las sesiones llamadas de coordinación, cada delegado presenta, ante el equipo de coordinación correspondiente, una valuación fundamentada de la solución de cada uno de sus estudiantes. El equipo de coordinación del problema en cuestión determina la calificación respectiva.

Se otorgan al menos 16 primeros lugares, alrededor de 32 segundos lugares y alrededor de 48 terceros lugares (para constituir, aproximadamente, la mitad de participantes premiados). Se otorgan además menciones honoríficas a los alumnos que no obtuvieron un primer, segundo o tercer lugar, pero que obtuvieron el máximo puntaje de 7 puntos en al menos un problema del examen

Se pueden otorgar premios especiales a aquellas soluciones presentadas por los alumnos en algún problema del examen si, a juicio del Tribunal de Coordinación, estas son muy sobresalientes. Dentro del Concurso Nacional se selecciona también un grupo de al menos 8 alumnos más jóvenes, candidatos a participar en la Olimpiada Centroamericana y del Caribe del año siguiente. Asimismo, el estado sede del Concurso Nacional entrega el premio de la Copa Superación del año a la delegación que muestre mayor progreso relativo.

A partir del concurso nacional del 2013, se premia a un grupo de a lo más 8 mujeres, candidatas a representar a México en la Olimpiada Europea Femenil de Matemáticas. México participó por primera vez en esta competencia en el 2014.

En etapas posteriores se entrena y elige, de entre los alumnos ganadores enel Concurso Nacional a quienes integrarán las delegaciones que el siguiente año representarán a México en
\begin{itemize}
    \item la Olimpiada Internacional de Matemáticas,
    \item la Olimpiada Iberoamericana de Matemáticas,
    \item la Olimpiada Matemática de la Cuenca del Pacífico,
    \item la Olimpiada Matemática de Centroamérica y el Caribe,
    \item la Olimpiada Europea Femenil de Matemáticas,
    \item la Competencia Rumana de Campeones\footnote{No siempre se participa en esta competencia.} y
    \item la Olimpiada Iraní de Geometría.
\end{itemize}

\subsection{Entrenamientos de las preselecciones}
Los alumnos ganadores en el Concurso Nacional reciben entrenamientos intensivos. Se trabajan conceptos especiales que no se estudian generalmente en los sistemas preuniversitarios: Álgebra, Combinatoria, Geometría (Euclidiana y Vectorial) y Teoría de Números, haciendo énfasis en la resolución de problemas. Los instructores son profesores de varias universidades del país familiarizados con el tipo de problemas matemáticos que se trabajan en las olimpiadas y alumnos exolímpicos destacados que han continuado su preparación en matemáticas.

Los entrenamientos se llevan a cabo en distintos lugares del país durante 10 días cada 6 semanas, iniciando en diciembre y hasta el momento de participación en el respectivo concurso internacional. Los exámenes definitivos para seleccionar a las delegaciones que representan a México en la Olimpiada Europea Femenil se llevan a cabo en marzo; para la Olimpiada Internacional y la Olimpiada Centroamericana y del Caribe, en mayo, y para seleccionar a la delegación que representa a nuestro país en la Olimpiada Iberoamericana, en agosto.

Los gastos de entrenamiento son cubiertos por las instituciones patrocinadoras de la Olimpiada Mexicana de Matemáticas, a través del Comité Organizador de la misma. Los gastos de traslado de cada alumno al lugar donde se realiza el entrenamiento son cubiertos por el comité estatal de donde el alumno proviene.