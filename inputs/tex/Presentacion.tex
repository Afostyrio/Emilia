La Olimpiada Mexicana de Matemáticas es la competencia anual de matemáticas para estudiantes preuniversitarios más importante en nuestro país. Su objetivo es promover el estudio de las matemáticas en forma creativa, alejándose del estudio tradicional que promueve la memorización y mecanización, y buscando desarrollar el razonamiento y la imaginación de los jóvenes.

Anualmente cada Estado del país lleva a cabo, en forma autónoma, su Concurso Estatal y la preparación del equipo que lo representará en el Concurso Nacional. A este concurso asisten 6 alumnos de cada Estado, dando un total de 192 participantes. Además, asisten uno o dos profesores por cada delegación estatal. Este evento se desarrolla en el mes de noviembre en algún Estado del país, mismo que patrocina fuertemente el evento. Asiste también un equipo de 25 personas que integran el Tribunal de Coordinación, encargado de calificar los exámenes presentados por los alumnos concursantes. Este equipo está formado por prestigiados profesores de todo el país y por alumnos que destacaron en olimpiadas anteriores y que han continuado su preparación en matemáticas.

Los alumnos con mejores calificaciones en el Concurso Nacional constituyen la preselección nacional, la cual recibe entrenamientos especiales durante varios meses. De esta preselección se eligen las delegaciones que representarán a México en las olimpiadas internacionales del año siguiente: Internacional, Iberoamericana, Centroamericana y del Caribe, de la Cuenca del Pacífico y en la Europea Femenil.

La participación de los alumnos en todos los concursos y entrenamientos es gratuita. Los gastos de viajes y alimentación son patrocinados por diversas instituciones, a través de la Sociedad Matemática Mexicana, institución organizadora de la Olimpiada a nivel nacional.

Para fortalecer el programa de la Olimpiada Mexicana de Matemáticas, el Comité Organizador de la misma realiza exámenes de práctica, cursos especiales para profesores y la publicación de material académico y de difusión. De manera general, este comité enlaza las inquietudes de los comités estatales, los alumnos participantes y la Sociedad Matemática Mexicana. Establece los contactos necesarios a nivel nacional y internacional para inscribir a las delegaciones que representan al país en los distintos concursos internacionales. Tramita los apoyos de las instituciones financiadoras de la OMM y maneja el presupuesto. Además, vigila la correcta aplicación del reglamento de la OMM.

El esfuerzo de un gran número de personas que han trabajado en el programa de la Olimpiada Mexicana de Matemáticas se ha visto recompensado por el papel destacado que ha tenido nuestro país a nivel internacional. Sobre todo es importante señalar el impacto en el ambiente educativo de nuestro país: muchos profesores y alumnos que se han acercado en algún momento a este programa han creado, de manera espontánea y altruista, innumerables talleres de resolución de problemas de matemáticas en los cuales han vertido sus experiencias. Asimismo, las universidades involucradas en la organización de las Olimpiadas de Matemáticas han recibido el fruto de su apoyo con el ingreso de alumnos con una excelente formación matemática obtenida durante los entrenamientos, los concursos y los que les ha ofrecido el programa de la olimpiada.