\section{México en las Olimpiadas internacionales}
México participa en siete eventos internacionales:
\begin{enumerate}
    \item Olimpiada Internacional de Matemáticas (IMO)
    \item Olimpiada Iberoamericana de Matemáticas (OIM)
    \item Olimpiada Matemática de Centroamérica y el Caribe (OMCC)
    \item Olimpiada Matemática de la Cuenca del Pacífico (APMO)
    \item Olimpiada Europea Femenil de Matemáticas (EGMO)
    \item Rumana de Campeones (RMM)
    \item Olimpiada Iraní de Geometría (IGO)
\end{enumerate}
\section{Descripción de los distintos concursos}
\subsection{Olimpiada Internacional de Matemáticas}
La Olimpiada Internacional de Matemáticas (IMO) se celebra en el mes de julio. Pueden asistir un máximo de 6 alumnos y dos profesores por país. Uno de los profesores, llamado líder de la Delegación, forma parte del Jurado. El Jurado se reúne unos días antes de iniciar el concurso para discutir la selección de los problemas que formarán el examen, la redacción de los enunciados de dichos problemas y la traducción de los mismos al idioma natural de cada país. La selección se hace a partir de un banco de problemas que elabora cuidadosamente un grupo de profesores durante los meses previos al concurso y que han sido extraídos de colaboraciones de todos los países participantes.

Se aplica un examen que consta de 2 pruebas escritas en dos días consecutivos, cada una de las cuales consta de tres problemas de matemáticas. Cada concursante presenta por escrito su solución para cada uno de los seis problemas.

El otro profesor de cada país, llamado tutor, llega al iniciar el concurso internacional, acompañando a los alumnos de su delegación. Durante los días del examen, permanece alojado junto con sus alumnos y lejos (en general, en ciudades distintas) del líder de la delegación. Al terminar el examen, los dos profesores de cada delegación se reúnen para calificar los exámenes de sus alumnos siguiendo los criterios acordados en una reunión previa (cada problema es calificado con un número entero del 0 al 7). A su vez, el Tribunal de Coordinación, compuesto por distinguidos miembros de la comunidad matemática internacional, revisa también los exámenes. Cada jefe de delegación presenta, ante el equipo de coordinación correspondiente, una evaluación fundamentada de la solución de cada uno de sus estudiantes. El equipo de coordinación del problema en cuestión determina la calificación respectiva.

Se otorgan medallas de oro, de plata y de bronce aproximadamente a la mitad de los participantes, distribuyéndolas en una razón aproximada de 1:2:3. Se otorgan menciones honoríficas a los alumnos que no ganaron una medalla pero que obtuvieron 7 puntos, la puntuación máxima, en al menos un problema del examen. Además, se pueden otorgar premios especiales a aquellas soluciones presentadas por los alumnos en algún problema del examen, si a juicio del Tribunal de Coordinación son muy sobresalientes.

\subsection{Olimpiada Iberoamericana de Matemáticas}
La Olimpiada Iberoamericana de Matemáticas (OIM) se celebra en el mes de septiembre. El sistema de competencia y evaluación se lleva a cabo con un esquema similar al de la Olimpiada Internacional y el Concurso Nacional. Pueden asistir un máximo de 4 alumnos y dos profesores por país. A partir de la V Olimpiada Iberoamericana se instituyó la Copa Puerto Rico, que se entrega cada año para reconocer al país que tiene mayor progreso relativo. Las reglas que se aplican para esta copa son similares a las que se aplican en México para otorgar la Copa Superación.

\subsection{Olimpiada Matemática de Centroamericana y el Caribe}
El sistema de competencia y evaluación de la Olimpiada Centroamericana y del Caribe (OMCC) se lleva a cabo con un esquema similar al de las comentadas anteriormente. El concurso centroamericano se realiza en el mes de junio y hasta el 2017 asistieron un máximo de 3 alumnos y dos profesores por país. A partir del 2018, asistirán un máximo de 4 alumnos y dos profesores por país.

La organización de esta Olimpiada consideró que Cuba, Colombia y México son los países más experimentados en olimpiadas de matemáticas; por esta ra- zón, se les impusieron algunas restricciones más fuertes de participación. En la I Olimpiada Centroamericana y del Caribe, México participó con una delegación de alumnos que radicaban en los estados del sur de la República. En 2000, la restricción para México consistió en formar su delegación con alumnos de todos los estados del país, pero un año más jóvenes que los alumnos de los otros países. A partir de 2001 las reglas de participación son las mismas para todos los países: no cumplir 17 años antes o durante el año de participación en ese concurso internacional.

México se impuso una restricción más al participar en esta olimpiada, la cual consiste en que los participantes mexicanos no hayan aún entrado a la preparatoria.

\subsection{Olimpiada Matemática de la Cuenca del Pacífico}
La olimpiada de la Cuenca del Pacifíco (APMO) se realiza en el mes de marzo y participan todos los alumnos que se encuentren en ese momento en los entrenamientos nacionales. Los exámenes son calificados en México y se envían los 10 mejores por correo al país organizador. A partir del promedio de puntajes y de la desviación estándar se definen los puntajes de oro, plata y bronce. Un país puede obtener a lo más una medalla de oro, dos de plata y cuatro de bronce.

\subsection{Olimpiada Europea Femenil de Matemáticas}
La Olimpiada Europea Femenil de Matemáticas (EGMO) nace en 2012 como una manera de estimular la participación femenil en olimpiadas de matemáticas, siguiendo el ejemplo de China que ya contaba con una olimpiada exclusiva para mujeres. El modelo de competencia de esta olimpiada es el mismo que el de la IMO, con la diferencia de que los equipos son de cuatro mujeres. A pesar de que la olimpiada es europea, es posible la participación de equipos no europeos por invitación. La primera EGMO fue llevada a cabo en Cambridge, Inglaterra en el 2012. La tercera edición se llevó a cabo en Antalya, Turquía en abril de 2014. Esta fue la primera participación mexicana en esta olimpiada.

\subsection{Rumana de Campeones}
La \textit{Romanian Master of Mathematics} (RMM) nace en 2009 como un evento en el cual puedan competir los mejores países del mundo. Se realiza en Bucarest, Rumania en la última semana de febrero. Para participar en este evento es necesario quedar entre los mejores 20 países en la Olimpiada Internacional de Matemáticas del año anterior. El formato de esta competencia es el mismo que el de la IMO a diferencia de que se puede participar con un equipo de 4 a 6 estudiantes.

\subsection{Olimpiada Iraní de Geometría}
La Olimpiada Iraní de Geometría (IGO) se realizó por primera ocasión en el 2014 en Irán y nace como una olimpiada en la que todos los problemas son de geometría, por ser esta una de las ciencias más antiguas de la humanidad. A partir del 2015 este se volvió un concurso internacional a distancia (similar, en este sentido, a la de la Cuenca del Pacífico). Se cuenta con tres niveles de participación: elemental, medio y avanzado, que corresponden a 1\textsuperscript{ro} y 2\textsubscript{do} de secundaria; 3\textsuperscript{ro} de secundaria y 1\textsuperscript{er} año de preparatoria; y 2\textsuperscript{do} y 3\textsuperscript{er} año de preparatoria, respectivamente.